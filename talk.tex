\input slides

\baselineskip=20pt plus 5pt
%\lineskip=5pt plus 3pt
%\lineskiplimit=5pt

\def\syntaxbg{.9 .9 .9}
\def\@numhskip{3ex}

\def\bheadline{\headline{blue}{white}{\TeX nical Programming}}
\def\bfootline{\footline{blue}{white}}
\def\gheadline{\headline{darkgreen}{white}{\TeX nical Programming}}
\def\gfootline{\footline{darkgreen}{white}}
\def\oheadline{\headline{orange}{white}{\TeX nical Programming}}
\def\ofootline{\footline{orange}{white}}

\def\tokcenterline{\line\bgroup\hfill\bgroup\aftergroup\hfill\aftergroup\egroup\let\_regU}

\let\epsilon=\varepsilon

\beginslide{[rgb]{.95 .95 1}}

    {\def\_buffer_height{10pt}
    \textbox[title]\framecoloredbox{0pt}{.5\vsize-.5\tboxh - 1cm}{\hsize-\frameboxbuf}{{blue}}
    \centerline{\localcolor{white}{\titlefont\fakebold{\TeX nical Programming}}}
    \endtextbox
    }
    
    \textbox\framecoloredbox{.5\hsize-.5\tboxw}{\p@title@ey+1cm}{10cm}{{blue}}
    \centerline{\color{white}Slurp}
    \centerline{\color{white}April, 2023}
    \endtextbox

\endslide

\beginslide
    \bheadline

    \textbox\empty{2cm}{.5\vsize-.5\tboxh}{15cm}{}
    {\fifteenbf\color{black}
        \blist
            \item \TeX{} engines and formats: \TeX{} vs \boldLaTeX
            \item Typesetting in plain \TeX
            \item Macros and primitives
            \item Repeated macros
            \item Lists
        \elist
    }
    \endtextbox

    \bfootline{Contents}
\endslide

\beginslide

    {\def\_buffer_height{10pt}
    \textbox\framecoloredbox{1cm}{.5\vsize-.5\tboxh}{\hsize-2cm}{{darkgreen}}
        \centerline{\headerfont\color{white}\fakebold{\TeX{} Engines and Formats}}
    \endtextbox
    }

\endslide

\beginslide

    \gheadline

    \textbox\empty{1cm}{2cm}{\hsize-2cm}{}
        \TeX{} refers mainly to three things:
        \blist
            \item The \TeX{} language, a typesetting language made up of {\fifteenit primitives\/} like \macroname\def{} and \macroname\hskip.
            \item The \TeX{} engine ({\tt tex}), a program for compiling files written in the \TeX{} language.
            \item The plain \TeX{} format, a bare-bones \TeX{} format.
        \elist
    \endtextbox

    \gfootline{\TeX{} engines and formats}

\endslide

\beginslide

    \gheadline

    \textbox\empty{1cm}{2cm}{\hsize-2cm}{}
        \TeX{} formats are self-contained macro packages whose intention it is to define macros for the user's use in creating documents.

        Some well-known formats are plain \TeX, \LaTeX, and Con\TeX t.

        Macro packages are not considered formats since they are not self-contained and their intention is to provide further abstraction on top of an existing format.

        The format can be specified during the invocation of the {\tt tex} command on the command line.
    \endtextbox

    \gfootline{\TeX{} engines and formats}

\endslide

\beginslide

    \gheadline

    \textbox\empty{1cm}{2cm}{\hsize-2cm}{}
        \TeX{} engines are programs which compile \TeX{} code into output.
        Historically this output was originally a {\tt dvi} file, but nowadays most output is a {\tt pdf} file.

        Some well-known formats are $\epsilon$-\TeX, pdf\TeX, Lua\TeX, and \XeTeX.

        Engines tend to add primitives to the base \TeX{} language: for example,
        \hbox{$\epsilon$-\TeX} adds the \macroname\numexpr{} and related primitives, pdf\TeX{} adds the \macroname\pdfliteral{} primitive among others.

        Note that invoking the program engine in the command line by itself also specifies the format: {\tt pdftex} runs the pdf\TeX engine with the plain \TeX{} format,
        while {\tt pdflatex} runs the pdf\TeX engine with the \LaTeX{} format.
    \endtextbox

    \gfootline{\TeX{} engines and formats}

\endslide

\beginslide

    {\def\_buffer_height{10pt}
    \textbox\framecoloredbox{1cm}{.5\vsize-.5\tboxh}{\hsize-2cm}{{darkgreen}}
        \centerline{\headerfont\color{white}\fakebold{Questions?}}
    \endtextbox
    }

\endslide

\beginslide

    {\def\_buffer_height{10pt}
    \textbox\framecoloredbox{1cm}{.5\vsize-.5\tboxh}{\hsize-2cm}{{blue}}
        \centerline{\headerfont\color{white}\fakebold{Typesetting in Plain \TeX}}
    \endtextbox
    }

\endslide

\beginslide
    \bheadline

    \textbox\empty{1cm}{2cm}{\hsize-2cm}{}
        \color{black}
        \TeX{} creates its pages by creating a vertical list and filling it with boxes, glue, leaders, penalties, kerns, etc.
        There are two types of boxes, horizontal boxes and vertical boxes.

        Boxes have width, height, and depth.
        The height of a box is visually its positive vertical displacement relative to the baseline, and its depth is its negative vertical displacement relative to the baseline.
        For example:

        \medskip
        \font\largerm=cmr10 at 40pt
        \centerline{\demobox{\largerm yelling}}
    \endtextbox

    \bfootline{Typesetting in plain \TeX}
\endslide

\beginslide
    \bheadline

    \textbox[hboxbox]\coloredbox{1cm}{2cm}{2cm}{{blue}}
        \centerline{\color{white}\macroname\hbox}
    \endtextbox

    \textbox[hboxexbox]\empty{4cm}{\p@hboxbox@oy}{\hsize - 5cm}{}%
        \macrousage\hbox to/bslash^^00spread <dimension>{<horizontal material>}/emacrousage: Creates \hfill\penalty-1000a horizontal box, if
        {\fifteenit to dimension\/} is specified then the width of the box is {\fifteenit dimension}.

        If {\fifteenit spread dimension\/} is specified then the width of the box is {\fifteenit dimension\/} more than its natural dimension.

        The height and depth of the box is equal to the height and depth of its contents.
    \endtextbox

    \textbox[vboxbox]\coloredbox{1cm}{\p@hboxexbox@ey + .5cm}{2cm}{{blue}}
        \centerline{\color{white}\macroname\vbox}
    \endtextbox

    \textbox\empty{4cm}{\p@vboxbox@oy}{\hsize - 5cm}{}%
        \macrousage\vbox to/bslash^^00spread <dimension>{<horizontal material>}/emacrousage: Creates \hfill\penalty-1000a vertical box, if
        {\fifteenit to dimension\/} is specified then the height of the box is {\fifteenit dimension}.

        If {\fifteenit spread dimension\/} is specified then the height of the box is {\fifteenit dimension\/} more than its natural dimension.

        The width of the box is the width of its contents, and the depth of the box is the depth of the final box in it.

        The box's baseline is the same as that of the final box inside, to align to the top box, use \macroname\vtop.
    \endtextbox

    \bfootline{Typesetting in plain \TeX}
\endslide

\beginslide
    \bheadline

    \textbox\empty{1cm}{2cm}{\hsize-2cm}{}
        Between two boxes there is something called {\fifteenit glue}, which connects the boxes.
        Glue is one type of blank space (the other being a kern).

        Glue has three attributes: its natural length, its maximum stretchiness, and its maximum shrinkage.
        Glue stretches and shrinks only when it needs to, and \TeX{} uses these attributes in order to fit material into widths the material couldn't properly fit into
        in its natural width.

        For example:

        \medskip
        \beginhi
\hbox to 5cm{hello\hskip 3cm plus 2cm}there
        \endhi
        \medskip

        Creates

        \medskip\centerline{\hbox to 5cm{hello\hskip3cm plus 2cm}there}\medskip

        Without the {\fifteentt plus2cm} we'd get the same output but with an overfull hbox warning.

        The amount of stretchiness and shrinkage can be infinite.

        All the dimensions may be negative as well.
    \endtextbox

    \bfootline{Typesetting in plain \TeX}
\endslide

\beginslide
    \bheadline

    \textbox[hskipbox]\coloredbox{1cm}{2cm}{2cm}{{blue}}
        \centerline{\color{white}\macroname\hskip}
    \endtextbox

    \textbox[hskipexbox]\empty{4cm}{\p@hskipbox@oy}{\hsize - 5cm}{}
        \macrousage\hskip <natural length> plus <stretch> minus <shrink>/emacrousage: Adds \hfill\penalty-1000horizontal glue with the specified natural length, maximal stretch and shrinkage.

        The stretch and shrink are optional.
    \endtextbox

    \textbox[vskipbox]\coloredbox{1cm}{\p@hskipexbox@ey + .5cm}{2cm}{{blue}}
        \centerline{\color{white}\macroname\vskip}
    \endtextbox

    \textbox[vskipexbox]\empty{4cm}{\p@vskipbox@oy}{\hsize - 5cm}{}
        \macrousage\vskip <natural length> plus <stretch> minus <shrink>/emacrousage: Adds \hfill\penalty-1000vertical glue with the specified natural length, maximal stretch and shrinkage.

        The stretch and shrink are optional.
    \endtextbox

    \textbox[kernbox]\coloredbox{1cm}{\p@vskipexbox@ey + .5cm}{2cm}{{blue}}
        \centerline{\color{white}\macroname\kern}
    \endtextbox

    \textbox\empty{4cm}{\p@kernbox@oy}{\hsize-5cm}{}
        \macrousage\kern <dimension>/emacrousage: Adds a kern whose dimension is {\fifteenit dimension}.
        Kerns, unlike glue are nonbreaking, nonstretching, and nonshrinking.
        The orientation of the kern (horizontal or vertical) is inferred by the context.
    \endtextbox

    \bfootline{Typesetting in plain \TeX}
\endslide

\beginslide
    \bheadline

    \textbox\empty{1cm}{2cm}{\hsize-2cm}{}
        \TeX{} has $3$ orders of infinities for glue stretching:
    \endtextbox

    \textbox\coloredbox{1cm}{3cm}{2cm}{{blue}}
        \centerline{\color{white}\tt fil}
    \endtextbox

    \textbox[filexbox]\empty{4cm}{3cm}{\hsize-5cm}{}
        First order {\tt fil}: \icode \hskip 0pt plus 1fil\relax/eicode{} creates glue\hfill\penalty-1000which has no natural length but has infinite stretchiness.
        A primitive version, \macroname\hfil, exists as well in place of the code above.
    \endtextbox

    \textbox\coloredbox{1cm}{\p@filexbox@ey + .5cm}{2cm}{{blue}}
        \centerline{\color{white}\tt fill}
    \endtextbox

    \textbox[fillexbox]\empty{4cm}{\p@filexbox@ey + .5cm}{\hsize-5cm}{}
        Second order {\tt fill}: \icode \hskip 0pt plus 1fill\relax/eicode{} creates glue which also has no natural length and infinite stretchiness.
        It takes precedent over first order infinities.
        A primitive version, \macroname\hfill, exists as well.
    \endtextbox

    \textbox\coloredbox{1cm}{\p@fillexbox@ey + .5cm}{2cm}{{blue}}
        \centerline{\color{white}\tt filll}
    \endtextbox

    \textbox[filllexbox]\empty{4cm}{\p@fillexbox@ey + .5cm}{\hsize-5cm}{}
        Third order {\tt filll}: Same as the other two, but takes precedent over both of them.
        No primitive version exists.
    \endtextbox

    \textbox\empty{1cm}{\p@filllexbox@ey + .5cm}{\hsize-2cm}{}
        Vertical versions of \macroname\hfil{} and \macroname\hfill{} exist, \macroname\vfil{} and \macroname\vfill.

        Another important primitive is \macroname\hss{} which can both shrink and stretch infinitely.
        It is analogous to \icode \hskip 0pt plus 1fil minus 1fil/eicode.

        It too has a vertical version \macroname\vss.
    \endtextbox

    \bfootline{Typesetting in plain \TeX}
\endslide

\beginslide
    \bheadline

    \textbox\_framebox{1cm}{2cm}{\hsize-2cm}{}
\beginhi
\def\line{\hbox to \hsize}
\def\centerline#1{\line{\hfil#1\hfil}
\def\rightline #1{\line{\hfil#1}}
\def\leftline  #1{\line{#1\hfil}
\def\rlap#1{\hbox to 0pt{#1\hss}}
\def\llap#1{\hbox to 0pt{\hss#1}}
\endhi
    \endtextbox

    \textbox\empty{1cm}{7cm}{\hsize-2cm}{}
        \macroname\line{} creates a box which spans the entire line.

        \macroname\centerline{} centers input relative to the line.

        \macroname\rightline{} and \macroname\leftline{} right and left-justify input respectively.

        \macroname\rlap{} typesets input and then seems to move back as if it hadn't been typeset.

        \macroname\llap{} moves back the width of its material and then typesets it.
    \endtextbox

    \bfootline{Typesetting in plain \TeX}
\endslide

\beginslide
    \bheadline

    \textbox\_framebox{1cm}{2cm}{\hsize-2cm}{}
\beginhi
\centerline{Centered Text}
\rightline{Right-Justified}
\leftline{Left-Justified}

\quitvmode\llap{outside}\hfill1\llap{0} and \rlap{1}0
    \hfill\rlap{outside}
\endhi
    \endtextbox

    \textbox\coloredbox{4cm}{7cm}{\hsize-8cm}{[rgb]{.9 .9 .9}}
        \centerline{Centered Text}
        \rightline{Right-Justified}
        \leftline{Left-Justified}
        
        \quitvmode\llap{outside}\hfill1\llap{0} and \rlap{1}0\hfill\rlap{outside}
    \endtextbox

    \bfootline{Typesetting in plain \TeX}
\endslide

\beginslide

    \bheadline

    \textbox\coloredbox{1cm}{2cm}{2cm}{{blue}}
        \centerline{\color{white}\macroname\hrule}
    \endtextbox

    \textbox\empty{4cm}{2cm}{\hsize-5cm}{}
        \macrousage\hrule width <dimen> height <dimen> depth <dimen>/emacrousage
        
        Creates a horizontal rule (a horizontal line).
        All of the dimensions are optional.
        If the width is not specified, then the width spans the width of the smallest box enclosing the rule.
        If the height is not specified, it is by default $0.4$pt.
        If the depth is not specified, it is by default $0$pt.

        This must be used in vertical mode (not horizontal mode, it is a horizontal rule because it is generally used to make
        horizontal lines in vertical mode).
    \endtextbox

    \textbox\coloredbox{1cm}{\lastey+.5cm}{2cm}{{blue}}
        \centerline{\color{white}\macroname\vrule}
    \endtextbox

    \textbox\empty{4cm}{\lastoy}{\hsize-5cm}{}
        \macrousage\vrule width <dimen> height <dimen> depth <dimen>/emacrousage
        
        Creates a vertical rule (a vertical line).
        All of the dimensions are optional.
        If the width is not specified, it is by default $0.4$pt.
        If the height or the depth are not specified, they span the height or depth of the smallest box enclosing the rule.

        This must be used in vertical mode.
    \endtextbox

    \bfootline{Typesetting in plain \TeX}
\endslide

\beginslide
    \bheadline

    \textbox\_framebox{1cm}{2cm}{\hsize-2cm}{}
\beginhi
\hrule
Hello \vrule

There \vrule \vbox to 20pt{}

\vbox{\line{}\hrule height 3pt}

\vskip.5cm
\hrule height 1pt
\vskip.5cm
\hrule height 2pt
\hrule height 2pt
\endhi
    \endtextbox

    \textbox\coloredbox{1cm}{10.5cm}{\hsize-2cm-\colorboxbuf}{[rgb]{.9 .9 .9}}
        \hrule
        Hello \vrule
        
        There \vrule \vbox to 20pt{}
        
        \vbox{\hbox to 5cm{}\hrule height 3pt}

        \vskip.5cm
        \hrule height 1pt
        \vskip.5cm
        \hrule height 2pt
        \hrule height 2pt
    \endtextbox

    \bfootline{Typesetting in plain \TeX}
\endslide

\beginslide
    \bheadline

    \textbox\_framebox{1cm}{2cm}{\hsize-2cm}{}
\beginhi
\def\boxed#1{\vbox{\hrule \hbox{\vrule #1\vrule}\hrule}}
\def\badbox#1{\hbox{\vrule \vbox{\hrule #1\hrule}\vrule}}

\boxed{Typesetting in plain \TeX}

\vskip.5cm
\badbox{Typesetting in plain \TeX}
\endhi
    \endtextbox

    \textbox\coloredbox{1cm}{7.5cm}{\hsize-2cm-\colorboxbuf}{[rgb]{.9 .9 .9}}
        \gdef\boxed#1{\vbox{\hrule \hbox{\vrule #1\vrule}\hrule}}
        \def\badbox#1{\hbox{\vrule \vbox{\hrule #1\hrule}\vrule}}
        
        \boxed{Typesetting in plain \TeX}

        \vskip.5cm
        \badbox{Typesetting in plain \TeX}
    \endtextbox

    \bfootline{Typesetting in plain \TeX}
\endslide

\beginslide
    \bheadline

    \textbox\_framebox{1cm}{2cm}{\hsize-2cm}{}
\beginhi
\def\spas{3pt}
\def\spacebox#1{%
    \hbox{%
        \vrule%
        \vbox{%
            \hrule \vskip\spas%
            \hbox{\hskip\spas #1\hskip\spas}%
            \vskip\spas \hrule%
        }%
        \vrule%
    }%
}

\spacebox{Typesetting in plain \TeX}
\endhi
    \endtextbox

    \textbox\coloredbox{1cm}{13cm}{\hsize-2cm-\colorboxbuf}{[rgb]{.9 .9 .9}}
        \def\spas{3pt}
        \def\spacebox#1{%
            \hbox{%
                \vrule%
                \vbox{%
                    \hrule \vskip\spas%
                    \hbox{\hskip\spas #1\hskip\spas}%
                    \vskip\spas \hrule%
                }%
                \vrule%
            }%
        }

        \quitvmode\spacebox{Typesetting in plain \TeX}
    \endtextbox

    \bfootline{Typesetting in plain \TeX}
\endslide

\beginslide
    \bheadline

    \textbox\empty{1cm}{2cm}{\hsize-2cm}{}
        \TeX{} has box registers which you can use to store boxes.
        The number of box registers depends on which engine you use (the original \TeX{} engine had $256$ from \icode \box0/eicode{}
        to \icode \box255/eicode).
    \endtextbox

    \textbox\coloredbox{1cm}{\lastey+.5cm}{2cm}{{blue}}
        \centerline{\color{white}\macroname\newbox}
    \endtextbox

    \textbox\empty{4cm}{\lastoy}{\hsize-5cm}{}
        \macrousage\newbox<control sequence>/emacrousage: Allocates a box and sets {\it control sequence} to reference this
        box.
    \endtextbox

    \textbox\coloredbox{1cm}{\lastey + .5cm}{2cm}{{blue}}
        \centerline{\color{white}\macroname\setbox}
    \endtextbox

    \textbox\empty{4cm}{\lastoy}{\hsize-5cm}{}
        \macrousage\setbox<box number> = <box>/emacrousage: Sets the {\it box number}-th box register to be equal to
        {\fifteentt box}.
    \endtextbox

    \textbox\coloredbox{1cm}{\lastey + .5cm}{2cm}{{blue}}
        \centerline{\color{white}\macroname\box}
    \endtextbox

    \textbox\empty{4cm}{\lastoy}{\hsize-5cm}{}
        \macrousage\box<box number>/emacrousage: uses the {\it box number} box register.
        After using this, the box register is emptied.
    \endtextbox

    \textbox\coloredbox{1cm}{\lastey + .5cm}{2cm}{{blue}}
        \centerline{\color{white}\macroname\copy}
    \endtextbox

    \textbox\empty{4cm}{\lastoy}{\hsize-5cm}{}
        \macrousage\copy<box number>/emacrousage: like \macroname\box{} but the register is not emptied after its use.
    \endtextbox

    \textbox\coloredbox{1cm}{\lastey + .5cm}{2cm}{{blue}}
        \centerline{\color{white}\macroname\unhbox}
    \endtextbox

    \textbox\empty{4cm}{\lastoy}{\hsize-5cm}{}
        \macrousage\unhbox<box number>/emacrousage: uses the {\it box number} box register and removes a level of boxing.

\beginhi
\setbox0=\hbox{A} \setbox1=\hbox{\unhbox0 B}
\endhi

        Makes \icode \box1/eicode{} equal to \icode \hbox{AB}/eicode.

        The box register must be a horizontal box (\macroname\hbox).
        For vertical boxes, use \macroname\unvbox.
        There also exist \macroname\unhcopy{} and \macroname\unvcopy.
    \endtextbox

    \bfootline{Typesetting in plain \TeX}
\endslide

\beginslide
    \bheadline

    \textbox\empty{1cm}{2cm}{\hsize-2cm}{}
        You can access the width, height, and depth of a box register using \macroname\wd, \macroname\ht, and \macroname\dp.
        The use being \macrousage \wd<box register>/emacrousage.

        In order to print a \TeX{} variable (like the dimensions of a box), you must prepend it with the primitive \macroname\the.
    \endtextbox

    \textbox\_framebox{1cm}{\lastey+.5cm}{\hsize-2cm}{}
\beginhi
\setbox0=\hbox{Typesetting in plain \TeX}
Width: \the\wd0, Height: \the\ht0, Depth: \the\dp0
\endhi
    \endtextbox

    \textbox\coloredbox{1cm}{\lastey+5pt}{\hsize-2cm-\colorboxbuf}{[rgb]{.9 .9 .9}}
        \setbox0=\hbox{Typesetting in plain \TeX}
        Width: \the\wd0, Height: \the\ht0, Depth: \the\dp0
    \endtextbox

    \textbox\empty{1cm}{\lastey+.5cm}{\hsize-2cm}{}
        Also, notice that something like

\beginhi
\hbox{A}B
\endhi

        Gives
    \endtextbox

    \textbox\coloredbox{1cm}{\lastey+5pt}{\hsize-2cm-\colorboxbuf}{[rgb]{.9 .9 .9}}
        \hbox{A}B
    \endtextbox

    \textbox\empty{1cm}{\lastey+5pt}{\hsize-2cm}{}
        This is because when \TeX{} boxes \icode \hbox{A}/eicode{} it is in vertical mode, and so the box is placed into the
        vertical list.
        Then when it reads {\tt B}, it enters horizontal mode, then boxes the horizontal material (which is just the {\tt B} here)
        and places that into the vertical list.
    \endtextbox

    \bfootline{Typesetting in plain \TeX}
\endslide

\beginslide
    \bheadline

    \textbox\empty{1cm}{1cm}{\hsize-2cm}{}
        So the vertical list looks like

\begincode
\vbox{
    \hbox{A}
    \hbox{B}
}
/endcode

        So what we'd want to do is exit vertical mode before the \macroname\hbox, so its added to the horizontal list instead
        of the vertical one.
        Ie. we want the vertical list to look like:

\begincode
\vbox{
    \hbox{\hbox{A}B}
}
/endcode
        In order to enter horizontal mode, we need to add horizontal material.

        We can do this by unhboxing a void register, since \macroname\unhbox{} adds horizontal material.

        \TeX{} provides a box register which is meant to be always void, \macroname\voidb@x, so you can do

\appendlist{catcodes}{{@}{0}}
\beginhi
\unhbox\voidb@x\hbox{A}B
\endhi
    \endtextbox

    \textbox\coloredbox{1cm}{\lastey+5pt}{\hsize-2cm-\colorboxbuf}{[rgb]{.9 .9 .9}}
        \unhbox\voidb@x\hbox{A}B
    \endtextbox

    \textbox\empty{1cm}{\lastey+5pt}{\hsize-2cm}{}
        \TeX{} also provides the macro \macroname\leavevmode{} which is short for this.

        pdf\TeX{} provides the primitive \macroname\quitvmode{} which achieves the same thing.
    \endtextbox

    \bfootline{Typesetting in plain \TeX}
\endslide

\beginslide
    \bheadline

    \textbox\_framebox{1cm}{1cm}{\hsize-2cm}{}
\beginhi
\def\presentbox#1{{%
    \setbox0=\hbox{#1}%
    \hbox{%
        \vbox{\tt%
            \hbox{Width: \the\wd0}%
            \hbox{Height: \the\ht0}%
            \hbox{Depth: \the\dp0}%
        }%
        \quad\boxed{%
            \rlap{#1}%
            \vrule width\wd0 height .5pt depth 0pt%
        }%
    }%
}}

\presentbox{Typesetting in plain \TeX}
\endhi
    \endtextbox

    \textbox\coloredbox{1cm}{\lastey+5pt}{\hsize-2cm-\colorboxbuf}{[rgb]{.9 .9 .9}}
        \def\presentbox#1{{%
            \setbox0=\hbox{#1}%
            \hbox{%
                \vbox{\tt%
                    \hbox{Width: \the\wd0}%
                    \hbox{Height: \the\ht0}%
                    \hbox{Depth: \the\dp0}%
                }%
                \quad\boxed{%
                    \rlap{#1}%
                    \vrule width\wd0 height .5pt depth 0pt%
                }%
            }%
        }}
        
        \quitvmode\presentbox{Typesetting in plain \TeX}
    \endtextbox

    \bfootline{Typesetting in plain \TeX}
\endslide

\beginslide
    \bheadline

    \textbox\_framebox{1cm}{2cm}{\hsize-2cm}{}
\beginhi
\def\centersym#1#2{\quitvmode{%
    \setbox0=\hbox{#1}%
    \rlap{#1}\hbox to \wd0{\hss#2\hss}%
}}

\centersym{$\bigcup$}{$\cdot$}
\endhi
    \endtextbox

    \textbox\coloredbox{1cm}{\lastey+5pt}{\hsize-2cm-\colorboxbuf}{[rgb]{.9 .9 .9}}
        \def\centersym#1#2{\quitvmode{%
            \setbox0=\hbox{#1}%
            \rlap{#1}\hbox to \wd0{\hss#2\hss}%
        }}

        \centersym{$\bigcup$}{$\cdot$}
    \endtextbox

    \bfootline{Typesetting in plain \TeX}
\endslide

\beginslide
    \bheadline

    \textbox\empty{1cm}{2cm}{\hsize-2cm}{}
        Leaders are a generalization of \TeX's concept of glue.
        The purpose of leaders are to {\fifteenit lead} your eyes across the page.

        This\leaders\hbox to 1em{\hss.\hss}\hfill is an example of leaders.

        The general use of leaders is

        \tokcenterline{\macrousage \leaders<box or rule><glue>/emacrousage}

        For example the above leaders was created by

\vskip5pt
\beginhi
\leaders\hbox to 1em{\hss.\hss}\hfill
\endhi

    Note that if a line ends with leaders or glue, it is removed (it doesn't really make sense to end a line with a blank space).
    So the following code:

\beginhi
\quitvmode\leaders\hbox to 1em{\hss.\hss}\hfill

Next line
\endhi

        yields:
    \endtextbox

    \textbox\coloredbox{1cm}{\lastey+5pt}{\hsize-2cm-\colorboxbuf}{[rgb]{.9 .9 .9}}
        \quitvmode\leaders\hbox to 1em{\hss.\hss}\hfill
        
        Next line
    \endtextbox

    \textbox\empty{1cm}{\lastey+5pt}{\hsize-2cm}{}
        There is no leaders here because the paragraph ends with leaders.
        In order to get around this you can simply place an empty box, or an empty kern after the leaders.
    \endtextbox

    \bfootline{Typesetting in plain \TeX}
\endslide

\beginslide
    \bheadline

    \textbox\empty{1cm}{2cm}{\hsize-2cm}{}
        Notice the \macroname\quitvmode{} before the \hfill\break\macroname\leaders{}, this is necessary because \macroname\leaders{} works in
        vertical mode as well, but its glue must be vertical glue (ie \macroname\vskip, \macroname\vfill, etc).
    \endtextbox

    \textbox\coloredbox{1cm}{\lastey+.5cm}{2.5cm}{{blue}}
        \centerline{\color{white}\macroname\leaders}
    \endtextbox

    \textbox\empty{4.5cm}{\lastoy}{\hsize-5.5cm}{}
        \macroname\leaders{} places leaders aligned with (what seems like) an infinite grid of boxes (either aligned with a row or
        column of the grid, depending on the horizontal/vertical context).

        This grid of boxes has the width/height of the box/rule in the leaders, and starts left aligned with the smallest enclosing
        box of the material.

        \macroname\leaders{} starts placing the leaders at the first box fully enclosed in the available space, and continues to
        the last available box (which is placed according to the input glue)

        So for example doing something like

{\let\_output_line_number=\@gobble
\beginhi
Hello\leaders\hbox to.5cm{\hss.\hss}\hfill there!
\endhi
}

        \quitvmode
        \rlap{\color{grey}%
            \line{\leaders\hbox to.5cm{\hss\boxed{\hbox to.5cm{\hss\vbox to.5cm{\vss}}}\hss}\hfil}%
        }%
        Hello\hfill there!

        So the first box is placed in the $4$th box (the first box which isn't occupied), and the last in the $4$th to last (the
        last box which isn't occupied).

        \quitvmode
        \rlap{\color{grey}%
            \line{\leaders\hbox to.5cm{\hss\boxed{\hbox to.5cm{\hss\vbox to.5cm{\vss}}}\hss}\hfill}%
        }%
        Hello\leaders\hbox to.5cm{\hss.\hss}\hfill there!
    \endtextbox

    \bfootline{Typesetting in plain \TeX}
\endslide

\beginslide
    \bheadline

    \textbox\coloredbox{1cm}{2cm}{2.5cm}{{blue}}
        \centerline{\color{white}\macroname\cleaders}
    \endtextbox

    \textbox\empty{4.5cm}{\lastoy}{\hsize-5.5cm}{}
        The process here is simpler, the boxes or rules are all packed tightly next to each other and an equal amount of glue is
        placed on either ends of the leaders:

        \quitvmode\rlap{leaders}\hphantom{cleaders}:\leaders\hbox to1cm{\hss.\hss}\hfill end

        cleaders:\cleaders\hbox to1cm{\hss.\hss}\hfill end
    \endtextbox

    \textbox\coloredbox{1cm}{\lastey+.5cm}{2.5cm}{{blue}}
        \centerline{\color{white}\macroname\xleaders}
    \endtextbox

    \textbox\empty{4.5cm}{\lastoy}{\hsize-5.5cm}{}
        The boxes or rules here are all spaced equally apart.
        If there are $q$ boxes/rules, then an equal amount of glue is placed in the $q+1$ spots between/around them.

        \quitvmode\rlap{leaders}\hphantom{xleaders}:\leaders\hbox to1cm{\hss.\hss}\hfill end

        xleaders:\xleaders\hbox to1cm{\hss.\hss}\hfill end
    \endtextbox

    \bfootline{Typesetting in plain \TeX}
\endslide

\beginslide
    \bheadline

    \textbox\empty{1cm}{2cm}{\hsize-2cm}{}
        Notice that while the placement of the boxes with \macroname\leaders{} is independent of the surrounding material, so
        multiple \macroname\leaders{} will have aligned boxes:

        \def\dotsleads{\leaders\hbox to1em{\hss.\hss}\hfill}
        \line{Typesetting in plain \TeX\dotsleads}
        \line{Typesetting in plain\dotsleads\TeX}
        \line{Typesetting in\dotsleads plain \TeX}
        \line{Typesetting\dotsleads in plain \TeX}
        \line{\dotsleads Typesetting in plain \TeX}

        While \macroname\cleaders{} and \macroname\xleaders{} both are:

        \vskip5pt
        \def\dotsleads{\cleaders\hbox to1em{\hss.\hss}\hfill}
        \line{Typesetting in plain \TeX\dotsleads}
        \line{Typesetting in plain\dotsleads\TeX}
        \line{Typesetting in\dotsleads plain \TeX}
        \line{Typesetting\dotsleads in plain \TeX}
        \line{\dotsleads Typesetting in plain \TeX}
    \endtextbox

    \bfootline{Typesetting in plain \TeX}
\endslide

\beginslide
    \bheadline

    \textbox\empty{1cm}{2cm}{\hsize-2cm}{}
        An important part of \TeX{} is the concept of {\fifteenit alignment}.

        In \LaTeX{} this takes the form of environments like {\tt tabular} and {\tt array}, but alignment is even more powerful.

        The main \TeX{} primitive which allows for alignment is \macroname\halign{}.

        The input of \macroname\halign{} comes in two parts: the preamble and the actual alignment material.
        The preamble of an alignment dictates how to align the material, it is simply a list of what material to place around the
        alignment material.
    \endtextbox

    \bfootline{Typesetting in plain \TeX}
\endslide

\beginslide
    \bheadline

    \textbox\_framebox{1cm}{2cm}{\hsize-2cm}{}
\beginhi
\tabskip=1cm
\halign{#\hfil&\hfil#\hfil&\hfil#\cr
    Left Aligned&Center Aligned&Right Aligned\cr
    Typesetting&in plain&\TeX\cr
}
\endhi
    \endtextbox

    \textbox\coloredbox{1cm}{\lastey+5pt}{\hsize-2cm-\colorboxbuf}{[rgb]{.9 .9 .9}}
        \tabskip=1cm
        \halign{#\hfil&\hfil#\hfil&\hfil#\cr
            Left Aligned&Center Aligned&Right Aligned\cr
            Typesetting&in plain&\TeX\cr
        }
    \endtextbox

    \textbox\empty{1cm}{\lastey+.5cm}{\hsize-2cm}{}
        Notice here that as opposed to \LaTeX{} where \macroname\\{} delimits the rows in an alignment, \macroname\halign{} uses
        \macroname\cr{} (carriage return).

        Here the preamble is \icode #\hfil&\hfil#\hfil&\hfil#/eicode{}

        This means for the first entry of each row, the template is \icode #\hfil/eicode{}, this right aligns it.
        What's significant about this is that the width of each entry is the maximum width of the all the entries in the column.

        Similarly the second and third templates center and left-align their entries respectively.

        \macroname\tabskip{} is the glue added between every column (as well as before the first and after the last column).
    \endtextbox

    \bfootline{Typesetting in plain \TeX}
\endslide

\beginslide
    \bheadline

    \textbox\empty{1cm}{2cm}{\hsize-2cm}{}
        \macroname\tabskip{} can also be altered within the preamble of an alignment.

        The glue inserted before the first column is equal to the value of \macroname\tabskip{} when \macroname\halign{} is called,
        and subsequent changes of \macroname\tabskip{} within the preamble of the alignment alter the glue inserted after the
        current and subsequent columns.
    \endtextbox

    \textbox\_framebox{1cm}{\lastey+.5cm}{\hsize-2cm}{}
\beginhi
\tabskip=0pt
\halign{#\hfil\tabskip=1cm&\hfil#\hfil&\hfil#\tabskip=0pt\cr
    Left Aligned&Center Aligned&Right Aligned\cr
    Typesetting&in plain&\TeX\cr
}
\endhi
    \endtextbox

    \textbox\coloredbox{1cm}{\lastey+5pt}{\hsize-2cm-\colorboxbuf}{[rgb]{.9 .9 .9}}
        \tabskip=0pt
        \halign{#\hfil\tabskip=1cm&\hfil#\hfil&\hfil#\tabskip=0pt\cr
            Left Aligned&Center Aligned&Right Aligned\cr
            Typesetting&in plain&\TeX\cr
        }
    \endtextbox

    \textbox\coloredbox{1cm}{\lastey+.5cm}{\hsize-2cm-\colorboxbuf}{[rgb]{.9 .9 .9}}
        \tabskip=1cm
        \halign{#\hfil&\hfil#\hfil&\hfil#\cr
            Left Aligned&Center Aligned&Right Aligned\cr
            Typesetting&in plain&\TeX\cr
        }
    \endtextbox

    \textbox\empty{1cm}{\lastey+5pt}{\hsize-2cm}{}
        For comparison, above is the output of the previous alignment.
    \endtextbox

    \bfootline{Typesetting in plain \TeX}
\endslide

\beginslide
    \bheadline

    \textbox\empty{1cm}{2cm}{\hsize-2cm}{}
        plain \TeX{} provides the macro \macroname\ialign{} which sets \macroname\tabskip{} to $0$ and calls \macroname\halign{}.

        We can also set the width of \macroname\halign{} similar to an \macroname\hbox{} via
        
        \tokcenterline{\macrousage\halign to <width>{...}/emacrousage{}}

        Instead of \macroname\cr, we can use \macroname\crcr{} which does the same thing, but if \macroname\crcr{} comes after
        another \macroname\cr, it does nothing.

        Thus we can create a macro which is similar to \LaTeX's {\tt align} environment
    \endtextbox

    \bfootline{Typesetting in plain \TeX}
\endslide

\beginslide
    \bheadline

    \textbox\_framebox{1cm}{.5\vsize-.5\tboxh}{\hsize-2cm}{}
\beginhi
\def\align#1{%
    \tabskip=0pt plus 1fil\relax%
    \halign to \hsize{%
        \hfil$\displaystyle##{}$\tabskip=0pt&%
        $\displaystyle{}##$\hfil\tabskip=0pt plus 1fil\cr%
        #1\crcr%
    }%
}

\align{
    \sum_{n=1}^\infty a_n &= 1\cr
    n! &= \prod_{i=1}^n i\cr
}
\endhi
    \endtextbox

    \bfootline{Typesetting in plain \TeX}
\endslide

\beginslide
    \bheadline

    \textbox\coloredbox{1cm}{.5\vsize-.5\tboxh}{\hsize-2cm-\colorboxbuf}{[rgb]{.9 .9 .9}}
        \def\align#1{%
            \tabskip=0pt plus 1fil\relax%
            \halign to \hsize{%
                \hfil$\displaystyle##{}$\tabskip=0pt&%
                $\displaystyle{}##$\hfil\tabskip=0pt plus 1fil\cr%
                #1\crcr%
            }%
        }

        \catcode`\_=8
        \align{
            \sum_{n=1}^\infty a_n &= 1\cr
            n! &= \prod_{i=1}^n i\cr
        }
    \endtextbox

    \bfootline{Typesetting in plain \TeX}
\endslide

\beginslide
    \bheadline

    \textbox\empty{1cm}{2cm}{\hsize-2cm}{}
        We can use the \macroname\openup{} macro to change the amount of glue added between lines (to ``open up'' the lines).
        This is used mostly in the context of alignment.
    \endtextbox

    \textbox\_framebox{1cm}{\lastey+5pt}{\hsize-2cm}{}
\beginhi
First paragraph

{\openup2\jot\halign{#\cr Hello\cr There\cr}}

Second paragraph

\halign{#\cr Hello\cr There\cr}

Third paragraph
\endhi
    \endtextbox

    \textbox\coloredbox{1cm}{\lastey+5pt}{\hsize-2cm}{[rgb]{.9 .9 .9}}
        First paragraph
        
        {\openup2\jot\halign{#\cr Hello\cr There\cr}}
        
        Second paragraph
        
        \halign{#\cr Hello\cr There\cr}
        
        Third paragraph
    \endtextbox

    \bfootline{Typesetting in plain \TeX}
\endslide

\beginslide
    \bheadline

    \textbox\empty{1cm}{2cm}{\hsize-2cm}{}
        \macroname\jot{} is simply a dimension (set equal to $3$pt by plain \TeX), it is customary to use \macroname\openup{} in
        terms of \macroname\jot s.

        \macroname\openup{} is cumulative: \icode \openup1\jot\openup-1\jot/eicode, has the same effect as if nothing had been
        done.

        \macroname\openup{} changes the amount of glue added between lines, so it is necessary to keep its changes local by
        placing it inside a group (\icode {...}/eicode).
    \endtextbox

    \bfootline{Typesetting in plain \TeX}
\endslide

\beginslide
    \bheadline

    \textbox\empty{1cm}{2cm}{\hsize-2cm}{}
        Tokens inside the preamble of an alignment are not expanded, unless they are preceded by the \macroname\span{} primitive, 
        which expands (ex-span-ds) the next token.

        \macroname\span{} has an entirely different meaning within the alignment material, it takes the place of {\tt\&} and
        merges the two entries into one box whose width matches with the width (including the inter-column glue) of the columns.

        Within an alignment entry \macroname\omit{} omits the current template, and instead the entry uses the trivial template
        {\tt\#}.

        Thus \macroname\span{} and \macroname\omit{} can be used in conjunction for cells which span multiple columns.

        Another useful primitive is \macrousage\noalign{<vertical material>}/emacrousage{} which can come after \macroname\cr{}
        and adds {\fifteenit vertical material\/} in place of interline glue.
    \endtextbox

    \bfootline{Typesetting in plain \TeX}
\endslide

\beginslide
    \bheadline

    \textbox\_framebox{1cm}{2cm}{\hsize-2cm}{}
\beginhi
{\tabskip=0pt
\openup1\jot
\halign{#\hfil\tabskip=5pt&\hfil#\tabskip=0pt\cr
\noalign{\hrule\vskip3pt}
\omit\span\omit\hfil Alignment Example\hfil\cr
\noalign{\vskip3pt\hrule\vskip\lineskip}
Typesetting&in\cr
plain&\TeX\cr}}
\endhi
    \endtextbox

    \textbox\coloredbox{1cm}{\lastey+5pt}{\hsize-2cm-\colorboxbuf}{[rgb]{.9 .9 .9}}
        {\tabskip=0pt
        \openup1\jot
        \halign{#\hfil\tabskip=5pt&\hfil#\tabskip=0pt\cr
        \noalign{\hrule\vskip3pt}
        \omit\span\omit\hfil Alignment Example\hfil\cr
        \noalign{\vskip3pt\hrule\vskip\lineskip}
        Typesetting&in\cr
        plain&\TeX\cr}}
    \endtextbox

    \bfootline{Typesetting in plain \TeX}
\endslide

\beginslide
    \bheadline

    \textbox\empty{1cm}{2cm}{\hsize-2cm}{}
        If you add a {\tt\&} before a template in an alignment's preamble, it is as if that template and all subsequent templates
        are repeated infinitely.

        So for example the preamble

        \begincode
\hfil#&&\hfil#&#\hfil
        /endcode

        Creates an alignment where the first template right-aligns, so does the next, the one after left-aligns, then
        right-aligns, then left-aligns, and so on ({\tt RRLRLRL...})

        Finally, there is another alignment primitive \macroname\valign{} which does its alignment vertically; instead of aligning
        by rows, it aligns by columns.

        The entries in \macroname\valign{} are in vertical mode.
        It is much less common than \macroname\halign{}.
    \endtextbox

    \bfootline{Typesetting in plain \TeX}
\endslide

\beginslide
    \bheadline

    \textbox\empty{1cm}{.9cm}{\hsize-2cm}{}
        Finally (just kidding, we have a lot more to go), let's bring together some of the concepts we've just covered and revisit
        our \macroname\spacebox{} macro.

        We can simplify it using alignment:
    \endtextbox

    \textbox\_framebox{1cm}{\lastey+5pt}{\hsize-2cm}{}
\beginhi
\def\spas{3pt}
\def\alignedbox#1{%
    \vbox{\offinterlineskip%
        \tabskip=0pt%
        \halign{##\tabskip=\spas&##&##\tabskip=0pt\cr
            \noalign{\hrule}
            \vrule&%
            \tabskip=\spas\valign{##\cr\hbox{#1}\cr}&%
            \vrule\cr
            \noalign{\hrule}
        }%
    }%
}

\alignedbox{Typesetting in plain \TeX}
\endhi
    \endtextbox

    \textbox\coloredbox{1cm}{\lastey+1pt}{\hsize-2cm-\colorboxbuf}{[rgb]{.9 .9 .9}}
        \def\spas{3pt}
        \def\alignedbox#1{%
            \vbox{\offinterlineskip%
                \tabskip=0pt%
                \halign{##\tabskip=\spas&##&##\tabskip=0pt\cr
                    \noalign{\hrule}
                    \vrule&\tabskip=\spas\valign{##\cr\hbox{#1}\cr}&\vrule\cr
                    \noalign{\hrule}
                }%
            }%
        }
        
        \quitvmode\alignedbox{Typesetting in plain \TeX}
    \endtextbox

    \bfootline{Typesetting in plain \TeX}
\endslide

\beginslide
    \bheadline

    \textbox\empty{1cm}{2cm}{\hsize-2cm}{}
        Just kidding, that wasn't simpler.

        But alignment is more versatile.
        Instead of boring rules, we will use leaders to create patterned lines, for example:
    \endtextbox

    \bfootline{Typesetting in plain \TeX}
\endslide

\beginslide
    \bheadline

    \textbox\_framebox{.5cm}{.5\vsize-.5\tboxh}{\hsize-1cm}{}
\beginhi
\def\hdotsline{\xleaders\hbox to 3pt{\hss.\hss}\hfil}
\def\vdotsline{\xleaders\vbox to 3pt{\vss\hbox{.}\vss}\vfil}
\def\dotbox#1{%
    \vbox{\offinterlineskip%
        \tabskip=0pt%
        \halign{##\cr
            \noalign{\hdotsline}
            \valign{##\cr
                \vdotsline\cr
                \hbox{#1}\cr
                \vdotsline\cr
            }\cr
            \noalign{\hdotsline}
        }%
    }%
}

\dotbox{Typesetting in plain \TeX}
\endhi
    \endtextbox

    \bfootline{Typesetting in plain \TeX}
\endslide

\beginslide
    \bheadline

    \textbox\coloredbox{1cm}{2cm}{\hsize-2cm-\colorboxbuf}{[rgb]{.9 .9 .9}}
        \def\squaredot{\vbox to 3pt{\vss\hbox to 3pt{\hss.\hss}\vss}}
        \def\hdotsline{\xleaders\squaredot\hfil}
        \def\vdotsline{\xleaders\squaredot\vfil}
        \def\dotbox#1{%
            \vbox{\offinterlineskip%
                \tabskip=0pt%
                \halign{##\cr
                    \hdotsline\cr
                    \valign{##\cr
                        \vdotsline\cr
                        \hbox{#1}\cr
                        \vdotsline\cr
                    }\cr
                    \hdotsline\cr
                }%
            }%
        }
        
        \quitvmode\dotbox{Typesetting in plain \TeX}
    \endtextbox

    \textbox\empty{1cm}{\lastey+.5cm}{\hsize-2cm}{}
        The primary use of \macroname\valign{} here is that it puts its entries in vertical mode whose height is the maximum
        height in the row.

        \macroname\dotbox{} doesn't add space around the text, doing so isn't terribly hard though.       
    \endtextbox

    \bfootline{Typesetting in plain \TeX}
\endslide

\beginslide
    \bheadline

    \textbox\empty{1cm}{2cm}{\hsize-2cm}{}
        \TeX{} utilizes many many registers to control the look of its output.
        Among them are some registers containing glue and dimensions which controls the space around paragraphs and lines.
    \endtextbox

    \textbox\coloredbox{1cm}{\lastey+.5cm}{3cm}{{blue}}
        \centerline{\color{white}\macroname\parindent}
    \endtextbox

    \textbox\empty{5cm}{\lastoy}{\hsize-6cm}{}
        The width of indentation before the first line of a paragraph.
        Plain \TeX{} sets this to $20$pt.
        \macroname\parindent{} is a dimension.
    \endtextbox

    \textbox\coloredbox{1cm}{\lastey+.5cm}{3cm}{{blue}}
        \centerline{\color{white}\macroname\parskip}
    \endtextbox

    \textbox\empty{5cm}{\lastoy}{\hsize-6cm}{}
        The inter-paragraph glue.
        Plain \TeX{} sets this to {\tt 0pt plus 1pt}.
    \endtextbox

    \textbox\coloredbox{1cm}{\lastey+.5cm}{3cm}{{blue}}
        \centerline{\color{white}\macroname\leftskip}
    \endtextbox

    \textbox\empty{5cm}{\lastoy}{\hsize-6cm}{}
        The glue added to the left of every line.
        Plain \TeX{} sets this to {\tt 0pt}.
    \endtextbox

    \textbox\coloredbox{1cm}{\lastey+.5cm}{3cm}{{blue}}
        \centerline{\color{white}\macroname\rightskip}
    \endtextbox

    \textbox\empty{5cm}{\lastoy}{\hsize-6cm}{}
        The glue added to the right of every line.
        Plain \TeX{} sets this to {\tt 0pt}.
    \endtextbox

    \textbox\empty{1cm}{\lastey+.5cm}{\hsize-2cm}{}
        For example, if we'd like to center a paragraph we could set the left and right glue to have infinite stretchability.
    \endtextbox

    \bfootline{Typesetting in plain \TeX}
\endslide

\beginslide
    \bheadline

    \textbox\_framebox{1cm}{1cm}{\hsize-2cm}{}
\beginhi
{\parindent=0pt
\leftskip=0pt plus 1fill \rightskip=0pt plus 1fill\relax
It is not the plan of this essay to discuss the
millennium-old problem of faith and reason. Theory is not
my concern at the moment. I want to instead focus attention
on a human-life situation in which the man of faith as an
individual concrete being, with his cares and hopes,
concerns and needs, joys and sad moments, is entangled.
\par
}
\endhi
    \endtextbox

    \textbox\coloredbox{1cm}{\lastey+5pt}{\hsize-2cm-\colorboxbuf}{[rgb]{.9 .9 .9}}
        {\parindent=0pt
        \leftskip=0pt plus 1fill \rightskip=0pt plus 1fill\relax
        It is not the plan of this essay to discuss the
        millennium-old problem of faith and reason. Theory is not
        my concern at the moment. I want to instead focus attention
        on a human-life situation in which the man of faith as an
        individual concrete being, with his cares and hopes,
        concerns and needs, joys and sad moments, is entangled.
        \par
        }
    \endtextbox

    \textbox\empty{1cm}{\lastey+.5cm}{\hsize-2cm}{}
        Notice that \macroname\par{} (which ends the current paragraph) must be in the same group as the paragraph.
        Otherwise the paragraph would be ended outside the group once \macroname\leftskip{} and \macroname\rightskip{} have
        already reverted back to their original values.
    \endtextbox

    \bfootline{Typesetting in plain \TeX}
\endslide

\beginslide
    \bheadline

    \textbox\empty{1cm}{2cm}{\hsize-2cm}{}
        \TeX{} also provides registers for further altering the shape of a paragraph with \macroname\hangindent{} and
        \macroname\hangafter{}.

        \macroname\hangindent{} is a dimension which specifies the dimension of the ``hanging indentation''.

        \macroname\hangafter{} is a number which specifies which lines will be indented.

        Suppose \macroname\hangafter{} is $n$, if $n\geq0$ then the indented lines are $n+1$, $n+2$, and so on until the end of
        the paragraph.

        If $n<0$ then the indented lines are $1$, $2$, \dots, $|n|$.
    \endtextbox

    \bfootline{Typesetting in plain \TeX}
\endslide

\beginslide
    \bheadline

    \textbox\_framebox{1cm}{2cm}{\hsize-2cm}{}
\beginhi
{\hangindent=1cm \hangafter=-2
Therefore, whatever I am going to say here has been derived
not from philosophical dialectics, abstract speculation, or
detached impersonal reflections, but from actual situations
and experiences with which I have been confronted. Indeed,
the term ``lecture'' also is, in this context, a misnomer.
\par}
\endhi
    \endtextbox

    \textbox\coloredbox{1cm}{\lastey+5pt}{\hsize-2cm-\colorboxbuf}{[rgb]{.9 .9 .9}}
        {\hangindent=1cm \hangafter=-2
        Therefore, whatever I am going to say here has been derived
        not from philosophical dialectics, abstract speculation, or
        detached impersonal reflections, but from actual situations
        and experiences with which I have been confronted. Indeed,
        the term ``lecture'' also is, in this context, a misnomer.
        \par}
    \endtextbox

    \bfootline{Typesetting in plain \TeX}
\endslide

\beginslide

    {\def\_buffer_height{10pt}
    \textbox\framecoloredbox{1cm}{.5\vsize-.5\tboxh}{\hsize-2cm}{{blue}}
        \centerline{\headerfont\color{white}\fakebold{Questions?}}
    \endtextbox
    }

\endslide

\beginslide

    {\def\_buffer_height{10pt}
    \textbox\framecoloredbox{1cm}{.5\vsize-.5\tboxh}{\hsize-2cm}{{orange}}
        \centerline{\headerfont\color{white}\fakebold{Macros and Primitives}}
    \endtextbox
    }

\endslide

\beginslide
    \oheadline

    \textbox\empty{1cm}{2cm}{\hsize-2cm}{}
        While many modern languages provide some form of functions (or procedures, or subroutines), \TeX{} does not.
        Instead \TeX{} is a macro-based language.

        This makes sense since \TeX's purpose is to typeset, not program.
        A macro is simply something which takes input and swaps it with some output.

        Some other languages, like C, also have macros.
        For example

{\bgroup\lccode`\?=`\# \lowercase{\egroup\appendlist{initcolors}{{?}{\_nextother}{.5 0 0}}}
\beginhi
#define MACRO(a) (Input is a)
MACRO(hello)
\endhi
}

        Will create a file containing the line {\tt Input is hello} if the C preprocessor is run on it.

        \TeX{} macros are similar: they take input and swap it with some output within the document.
        For example

\beginhi
\def\macro#1{Input is #1}
\macro{hello}
\endhi

        Will create output with the line {\tt Input is hello} once it is compiled.

        But this is a simple example; \TeX{} macros are far more powerful than C's.
        In fact, \TeX{} is Turing Complete, meaning any algorithm can be written using \TeX{}.
    \endtextbox

    \ofootline{Macros and primitives}
\endslide

\beginslide
    \oheadline

    \textbox\empty{1cm}{2cm}{\hsize-2cm}{}
        \TeX{} macros are marked by the use of a {\fifteenit escape character}, which is usually the backslash
        {\fifteentt\char"5C}.

        Calling these {\fifteenit macros} is actually incorrect, what follows an escape character is a {\fifteenit control
        sequence}.
        There are two types of control sequences: a {\fifteenit control word} which is a sequence of {\fifteenit letters} followed
        by a non-letter, for example \macroname\sqrt.
        And a {\fifteenit control letter} is a control sequence consisting of a single non-letter, for example \macroname\".

        A critical difference between control words and letters is that \TeX{} will always ignore a space after a control word (can you
        think of why?) while it will not ignore a space after a control letter.

        Some of \TeX's control sequences are {\fifteenit primitives}, these are control sequences which are not decomposable into
        simpler functions.
        \TeX's primitives form the backbone on which the rest of \TeX{} is defined.
        Some examples of primitives are \macroname\input{} and \macroname\hbox.

        You can define your own control sequences, called {\fifteenit macros} using primitives like \macroname\def{}.
        This section will focus on this.
    \endtextbox

    \ofootline{Macros and primitives}
\endslide

\beginslide
    \oheadline

    \textbox\empty{1cm}{2cm}{\hsize-2cm}{}
        The general usage of the \macroname\def{} primitive is:

\vskip5pt
\tokcenterline{\macrousage
\def<control sequence><parameter text>{<replacement text>}
/emacrousage}
\vskip5pt

        Which defines a control sequence which matches text to {\fifteenit parameter text} and replaces it with the
        {\fifteenit replacement text}.
        A simple example is as follows:

    \endtextbox

    \textbox\_framebox{1cm}{\lastey+.5cm}{\hsize-2cm}{}
\beginhi
\def\silly ABC{abc}

\silly ABCDEFG
\endhi
    \endtextbox

    \textbox\coloredbox{1cm}{\lastey+5pt}{\hsize-2cm-\colorboxbuf}{[rgb]{.9 .9 .9}}
        \def\silly ABC{abc}
        
        \silly ABCDEFG
    \endtextbox

    \ofootline{Macros and primitives}
\endslide

\beginslide
    \oheadline

    \textbox\empty{1cm}{2cm}{\hsize-2cm}{}
        Within the {\tt parameter text} of a macro we can define parameters using the special character {\tt\#}.
        A parameter is denoted by {\tt\#N} where {\tt N} is a number between $1$ and $9$, and the parameters must be ordered sequentially.

        The parameter text of a macro is matched lazily, ie. the first instance of a pattern will be matched.
        For example:
    \endtextbox

    \textbox\_framebox{1cm}{\lastey+.5cm}{\hsize-2cm}{}
\beginhi
\def\sillier A#1C{a(#1)c}

\sillier ABCABC
\endhi
    \endtextbox

    \textbox\coloredbox{1cm}{\lastey+5pt}{\hsize-2cm-\colorboxbuf}{[rgb]{.9 .9 .9}}
        \def\sillier A#1C{a(#1)c}
        
        \sillier ABCABC
    \endtextbox

    \ofootline{Macros and primitives}
\endslide

\beginslide
    \oheadline

    \textbox\empty{1cm}{1.5cm}{\hsize-2cm}{}
        When pattern-matching for a macro, \TeX{} will consider groups of tokens contained within braces (\icode {...}/eicode) to be a
        single atomic unit, and will not match it or the tokens contained within it to a non-parameter.

        If no delimiter follows a parameter in the parameter text, the parameter is whatever the next group of tokens is.
        That is, if the next tokens are contained within braces: \icode {...}/eicode, then the whole group is taken as the parameter
        (without the braces), and if it's a single token then that token is taken.

        For example:
    \endtextbox

    \textbox\_framebox{1cm}{\lastey+5pt}{\hsize-2cm}{}
\beginhi
\def\silliness#1{(#1)}
\def\sillier A#1C{a(#1)c}

\silliness{abc}def  % #1 = abc
                                 
\silliness abcdef   % #1 = a

\sillier AB{BC}C    % #1 = B{BC}
\endhi
    \endtextbox

    \textbox\coloredbox{1cm}{\lastey+5pt}{\hsize-2cm-\colorboxbuf}{[rgb]{.9 .9 .9}}
        \def\silliness#1{(#1)}
        \def\sillier A#1C{a(#1)c}
        
        \silliness{abc}def

        \silliness abcdef

        \sillier AB{BC}C
    \endtextbox

    \ofootline{Macros and primitives}
\endslide

\beginslide
    \oheadline

    \textbox\empty{1cm}{2cm}{\hsize-2cm}{}
        A small note to remember is that while:
        \begincode
            \def\macro#1#2{...}
            \def\macro #1#2{...}
        /endcode
        are equivalent,
        \begincode
            \def\macro#1#2{...}
            \def\macro #1 #2 {...}
        /endcode
        are not.
        The second \macroname\macro's parameters must be delimited with spaces, while the first's are not.
    \endtextbox

    \ofootline{Macros and primitives}
\endslide

\beginslide
    \oheadline

    \textbox\empty{1cm}{2cm}{\hsize-2cm}{}
        Before we get to writing some funky macros, let's first discuss a handful of very useful \TeX{} primitives.
    \endtextbox

    \textbox\coloredbox{1cm}{\lastey+.5cm}{3.5cm}{{orange}}
        \centerline{\color{white}\macroname\expandafter}
    \endtextbox

    \textbox\empty{5cm}{\lastoy}{\hsize-6cm}{}
        \macrousage\expandafter<token1><token2>/emacrousage: \TeX{} ignores {\it token1\/} at first, expands {\it token2\/} (once) and then
        places {\it token1\/} before the expansion.
    \endtextbox

    \textbox\coloredbox{1cm}{\lastey+.5cm}{3.5cm}{{orange}}
        \centerline{\color{white}\macroname\csname}
        \centerline{\color{white}\macroname\endcsname}
    \endtextbox

    \textbox\empty{5cm}{\lastoy}{\hsize-6cm}{}
        \macrousage\csname...\endcsname/emacrousage: \TeX{} reads the tokens between\hfill\penalty-1000
        \macroname\csname{} and \macroname\endcsname{} and totally expands them.
        After this expansion, only ASCII must remain and \TeX{} replaces this with a control sequence token whose name is this expansion.
        If this control sequence is undefined currently, this control sequence token is defined to be \macroname\relax.
    \endtextbox

    \textbox\coloredbox{1cm}{\lastey+.5cm}{3.5cm}{{orange}}
        \centerline{\color{white}\macroname\string}
    \endtextbox

    \textbox\empty{5cm}{\lastoy}{\hsize-6cm}{}
        \macrousage\string<token>/emacrousage: \TeX{} reads {\it token\/} without expanding it, and if it is a control sequence, it converts
        it to a token list of characters (of category {\tt other} other than spaces) which corresponds to its name, following the escape
        character token.
        If {\it token} is a character, it is preserved.
    \endtextbox

    \ofootline{Macros and primitives}
\endslide

\beginslide
    \oheadline

    \textbox\_framebox{1cm}{2cm}{\hsize-2cm}{}
\beginhi
\def\gobble#1{}
\def\strip{\expandafter\gobble\string}

\def\initstring#1#2{%
    \expandafter\edef\csname s@#1 \endcsname{#2}}   % Why is \expandafter necessary?
\def\printstring#1{\csname s@#1 \endcsname}
\def\appendstring#1#2{%
    \expandafter\edef\csname s@#1 \endcsname{\printstring{#1}\printstring{#2}}}

\initstring{first string}{Hello}
\initstring{second string}{ World!}
\appendstring{first string}{second string}
\printstring{first string}
\endhi
    \endtextbox

    \textbox\coloredbox{1cm}{\lastey+5pt}{\hsize-2cm}{[rgb]{.9 .9 .9}}
        \def\gobble#1{}
        \def\strip{\expandafter\gobble\string}
        
        \def\initstring#1#2{%
            \expandafter\edef\csname s@#1 \endcsname{#2}}
        \def\printstring#1{\csname s@#1 \endcsname}
        \def\appendstring#1#2{%
        	\expandafter\edef\csname s@#1 \endcsname{\printstring{#1}\printstring{#2}}}
        
        \initstring{first string}{Hello}
        \initstring{second string}{ World!}
        \appendstring{first string}{second string}
        \printstring{first string}
    \endtextbox

    \ofootline{Macros and primitives}
\endslide

\beginslide
    \oheadline

    \textbox\empty{1cm}{2cm}{\hsize-2cm}{}
        As you may be aware, certain characters in \TeX{} are assigned special purposes: curly braces (\icode{...}/eicode) begin and end groups, dollar signs start and end math mode,
        tildes (\icode~/eicode) create non-breaking spaces, etc.

        While it may seem like this is the case, it is not.
        No character is assigned a special purpose, as this implementation would not be robust: there are many times when you may want a character to have a different purpose (eg. verbatim environments).
        Instead each character is assigned a {\it category code} (catcode) and each category code has a special purpose.

        Thus while certain characters may have special attributes, they are not inherent to the character: they are inherent the catcode assigned to the character.
    \endtextbox

    \ofootline{Macros and primitives}
\endslide

\beginslide
    \oheadline

    \textbox\empty{1cm}{2cm}{\hsize-2cm}{}
        The following is a list of all category codes, their purpose, and ``canonical'' examples of characters with those category codes.
    \endtextbox

    \textbox\coloredbox{1cm}{\lastey+.5cm}{1cm}{{orange}}
        \centerline{\color{white}\tt0}
    \endtextbox

    \textbox\empty{3cm}{\lastoy}{\hsize-4cm}{}
        Escape character (eg. \icode\/eicode)

        An escape character denotes the beginning of a control sequence, which is an escape character (catcode $0$) followed by a string of letters (catcode $11$) or a single non-letter.
    \endtextbox

    \textbox\coloredbox{1cm}{\lastey+.25cm}{1cm}{{orange}}
        \centerline{\color{white}\tt1+2}
    \endtextbox

    \textbox\empty{3cm}{\lastoy}{\hsize-4cm}{}
        Begin (catcode $1$) and end (catcode $2$) group (eg. \icode{/eicode{} and \icode}/eicode)

        These characters begin and end nested groups, which are used for locality and parameters.
    \endtextbox

    \textbox\coloredbox{1cm}{\lastey+.25cm}{1cm}{{orange}}
        \centerline{\color{white}\tt3}
    \endtextbox

    \textbox\empty{3cm}{\lastoy}{\hsize-4cm}{}
        Math shift (eg. \icode$/eicode) %$

        This begins or ends math (or display math) mode.
    \endtextbox

    \textbox\coloredbox{1cm}{\lastey+.25cm}{1cm}{{orange}}
        \centerline{\color{white}\tt4}
    \endtextbox

    \textbox\empty{3cm}{\lastoy}{\hsize-4cm}{}
        Alignment tab (eg. \icode&/eicode)

        This is the character used in alignment primitives to begin new tabs.
    \endtextbox

    \textbox\coloredbox{1cm}{\lastey+.25cm}{1cm}{{orange}}
        \centerline{\color{white}\tt5}
    \endtextbox

    \textbox\empty{3cm}{\lastoy}{\hsize-4cm}{}
        End of line (eg. the carriage return/newline character)

        When \TeX{} encounters an end-of-line character, depending on its current state it may add a space or end the paragraph or nothing.
    \endtextbox

    \ofootline{Macros and primitives}
\endslide

\beginslide
    \oheadline

    \textbox\coloredbox{1cm}{2cm}{1cm}{{orange}}
        \centerline{\color{white}\tt6}
    \endtextbox

    \textbox\empty{3cm}{\lastoy}{\hsize-4cm}{}
        Parameter (eg. \icode #/eicode)

        This is the character used to denote parameters within macros.
    \endtextbox

    \textbox\coloredbox{1cm}{\lastey+.25cm}{1cm}{{orange}}
        \centerline{\color{white}\tt7+8}
    \endtextbox

    \textbox\empty{3cm}{\lastoy}{\hsize-4cm}{}
        Superscript (catcode $7$) and subscript (catcode $8$) (eg. \icode ^/eicode{} and \icode _/eicode)

        These are the characters used to create superscripts and subscripts in math mode.
    \endtextbox

    \textbox\coloredbox{1cm}{\lastey+.25cm}{1cm}{{orange}}
        \centerline{\color{white}\tt9}
    \endtextbox

    \textbox\empty{3cm}{\lastoy}{\hsize-4cm}{}
        Ignored (eg. NUL byte)

        These characters are ignored by \TeX.
    \endtextbox

    \textbox\coloredbox{1cm}{\lastey+.25cm}{1cm}{{orange}}
        \centerline{\color{white}\tt10}
    \endtextbox

    \textbox\empty{3cm}{\lastoy}{\hsize-4cm}{}
        Space (eg. the space character)

        Space characters make \TeX{} insert a space if it is not currently skipping spaces.
    \endtextbox

    \textbox\coloredbox{1cm}{\lastey+.25cm}{1cm}{{orange}}
        \centerline{\color{white}\tt11}
    \endtextbox

    \textbox\empty{3cm}{\lastoy}{\hsize-4cm}{}
        Letter (eg. a\dots z, A\dots Z)

        These are the characters which can be used in control words.
    \endtextbox

    \textbox\coloredbox{1cm}{\lastey+.25cm}{1cm}{{orange}}
        \centerline{\color{white}\tt12}
    \endtextbox

    \textbox\empty{3cm}{\lastoy}{\hsize-4cm}{}
        Other (eg. \icode @/eicode)

        An Other character is a character with no other category code.
        Its special purpose is that it has no special purpose.
        It is therefore useful for example when creating verbatim environments.
    \endtextbox

    \ofootline{Macros and primitives}
\endslide

\beginslide
    \oheadline

    \textbox\coloredbox{1cm}{2cm}{1cm}{{orange}}
        \centerline{\color{white}\tt13}
    \endtextbox

    \textbox\empty{3cm}{\lastoy}{\hsize-4cm}{}
        Active (eg. \icode ~/eicode)

        These are characters which act like macros, they can be defined with \macroname\def{} and \macroname\let.
    \endtextbox

    \textbox\coloredbox{1cm}{\lastey+.25cm}{1cm}{{orange}}
        \centerline{\color{white}\tt14}
    \endtextbox

    \textbox\empty{3cm}{\lastoy}{\hsize-4cm}{}
        Comment (eg. \icode %/eicode)

        Everything after this character until the end of the line (inclusive) are ignored.
    \endtextbox

    \textbox\coloredbox{1cm}{\lastey+.25cm}{1cm}{{orange}}
        \centerline{\color{white}\tt15}
    \endtextbox

    \textbox\empty{3cm}{\lastoy}{\hsize-4cm}{}
        Invalid (eg. DELETE)

        These characters are ignored but make \TeX{} print an error message.
    \endtextbox

    \ofootline{Macros and primitives}
\endslide

\beginslide
    \oheadline

    \textbox\empty{1cm}{2cm}{\hsize-2cm}{}
        Category codes can be accessed and altered using the \macroname\catcode{} primitive.
        \macroname\catcode{} should be followed by an integer corresponding to the character code of the character whose category code you'd like to access.
        \TeX{} uses ASCII codes for its character codes, but some unicode-supporting engines use unicode.

        For example since the ASCII value of `a' is $97$, doing \icode \catcode97=13/eicode{} changes `a` to be an active character.

        But remembering ASCII by heart is a pain, and fortunately not necessary.
        When \TeX{} is looking for a number and encounters a backtick ({\tt\char18}) followed by a character or a control sequence of one character, it uses the character code of that character.

        So for example we could have done \icode \catcode`a=13/eicode{} or \icode \catcode`\a=13/eicode.

        The reason for allowing control sequences as well is because a construct like \icode \catcode`\=12/eicode{} will not set \icode\/eicode's category $12$, rather it will set {\tt=}'s category code.
        But \icode \catcode`\\=12/eicode{} will work.
    \endtextbox

    \ofootline{Macros and primitives}
\endslide 

\beginslide
    \oheadline

    \textbox\empty{1cm}{2cm}{\hsize-2cm}{}
        The most common use of catcode manipulation is changing the catcode of {\tt @} (and to a lesser extent, {\tt\icode _/eicode}).
        This is commonly done by package authors to create internal macros which they want to hide from the user.
        This is done by changing the catcode of {\tt @} to $11$ (letter) and creating macros which have {\tt @} in their name.
        At the end of the package the catcode of {\tt @} is reverted to $12$ (other), its original value, and so users cannot use these macros without themselves changing the catcode of {\tt @}.

        This is done in \LaTeX{} with the macros \macroname\makeatletter{} and \macroname\makeatother.
        These essentially expand to \icode \catcode`@=11/eicode{} and \icode \catcode`@=12/eicode{} respectively.
    \endtextbox

    \ofootline{Macros and primitives}
\endslide

\beginslide
    \oheadline

    \textbox\empty{1cm}{2cm}{\hsize-2cm}{}
        Suppose you have a macro like
\let\_output_line_number=\@gobble
\def\syntax@buf{.25cm}
\beginhi
\def\makeatletter{\catcode`\@=11}
\endhi

        While this seems like a fine macro, it is actually problematic.
        For example suppose we have a macro called \macroname\version@name{} which contains a number, say $3$.
        Take a look at the following code:
\beginhi
\makeatletter
\version@name
\endhi

        This will actually fail with the error message

        \begincode
! Undefined control sequence 
l.X \makeatletter\version
                         @num
        /endcode

        This is because \TeX{} continues reading for numbers after the {\tt11} in \hfill\break
        \macroname\makeatletter, and so it comes across \icode\version@num/eicode.
        Since the category code of {\tt @} is still $12$, it reads this as the control sequence \macroname\version{} followed by {\tt @num}.
        So it attempts to expand \macroname\version{} to see if it contains a number, but since \macroname\version{} doesn't exist, it errors.

    \endtextbox

    \ofootline{Macros and primitives}
\endslide

\beginslide
    \oheadline

    \textbox\empty{1cm}{2cm}{\hsize-2cm}{}
        Now suppose it was called \macroname\versionname{} instead.
        This still errors, but for a different reason:

        \begincode
! Invalid code (113), should be in the range 0..15.
        /endcode

        If we follow the process above, notice that \TeX{} continues reading after the {\tt11}, encounters \macroname\versionnumber{} which it expands and finds a $3$.
        Thus it appends this $3$ to the current number it is reading to get $113$.
        So it attempts to set the category code of {\tt @} to $113$, which is an invalid category code.

        We can avoid all of these issues with a slight redefinition to our \macroname\makeatletter.
        These issues stem from the fact that \TeX{} continues to scan even after the $11$, so all we need to do is stop it from scanning past the $11$.
        This can be done by inserting a \macroname\relax{} after the $11$.
        \macroname\relax{} is a \TeX{} primitive which is unexpandable but does nothing, so it stops \TeX's scan without inserting anything into the stream.

\let\_output_line_number=\@gobble
\def\syntax@buf{.25cm}
\beginhi
\def\makeatletter{\catcode`@=11\relax}
\endhi

        In general with constructs where \TeX{} is scanning ahead to look for parameters, it is a good idea to place a \macroname\relax{} after the parameter to minimize the risk of issues like this.

    \endtextbox

    \ofootline{Macros and primitives}
\endslide

\beginslide
    \oheadline

    \textbox\empty{1cm}{2cm}{\hsize-2cm}{}
        Another common issue with catcodes is the issue of {\it frozen catcodes}.
        This issue stems from the fact that \TeX{} will freeze catcodes in certain situations, namely when parameterizing tokens and in the definition of a macro (the parameter and replacement text).

        Let's look at the following situation: suppose we have some macro called \macroname\m@cro{} and the following code:
    \endtextbox

    \textbox\_framebox{1cm}{\lastey+.25cm}{\hsize-2cm}{}
\beginhi
\def\identity#1{#1}
\identity{{\catcode`@=11\relax \m@cro}}

\def\macro{{\catcode`@=11\relax \m@cro}}
\macro
\endhi
    \endtextbox

    \textbox\empty{1cm}{\lastey+.25cm}{\hsize-2cm}{}
        The double braces here are in order to keep catcode alterations local.

        Both of these calls will fail due to catcode freezing.
        The first one fails because the catcode of {\tt @} in \macroname\m@cro{} is $12$ (other) when the parameter is scanned, and so {\tt @}'s catcode is frozen at $12$ and so \TeX{} sees
        \macroname\m@cro{} as the control sequence \macroname\m{} followed by {\tt @cro}.
        Since \macroname\m{} is not defined, it will error.

        Similarly, \macroname\macro{} fails because when the definition of \macroname\macro{} is read, {\tt @} has catcode $12$ (other) and so when \TeX{} tokenizes the replacement text of \macroname\macro,
        the catcode of {\tt @} is frozen at $12$.
    \endtextbox

    \ofootline{Macros and primitives}
\endslide

\beginslide
    \oheadline

    \textbox\empty{1cm}{2cm}{\hsize-2cm}{}
        \TeX{} also allows you to use \icode ^^/eicode{} to insert ASCII characters by following the \icode ^^/eicode{} with two lowercase hexadecimal digits.
        So for example \icode ^^61/eicode{} inserts the character whose ASCII value is $61$ in hex (or $97$ in decimal: ^^61)

        There is another convention where you follow \icode ^^/eicode{} with a single character and if that characters internal code is between $64$ and $127$ then $64$ is subtracted from the code.
        Otherwise $64$ is added to the code.
        So for example \icode ^^M/eicode{} gives ASCII character $77-64=13$ which is the carriage return character.
    \endtextbox

    \textbox\_framebox{1cm}{\lastey+.25cm}{\hsize-2cm}{}
\beginhi
{
    \catcode`^^0d=12 % Why are comments necessary?
    \catcode`@=11 %
    \gdef\@getline#1#2^^0d{\egroup#1{#2}}% ^^0d = \n
    \gdef\getline#1{\bgroup\catcode`^^0d=12\relax\@getline{#1}}%
}

\def\parenthesize#1{(#1)}
\getline\parenthesize Hello there!
\endhi
    \endtextbox

    \textbox\coloredbox{1cm}{\lastey+5pt}{\hsize-2cm-\colorboxbuf}{[rgb]{.9 .9 .9}}
        {
            \catcode`\^^0d=12 %
            \catcode`@=11 %
            \long\gdef\@getlineA#1#2^^0d{\egroup #1{#2}}%
            \gdef\getlineA#1{\bgroup\catcode`\^^0d=12 \@getlineA{#1}}%
        }
        
        \def\parenthesize#1{(#1)}
        \getlineA\parenthesize Hello there!
    \endtextbox

    \ofootline{Macros and primitives}
\endslide

\beginslide
    \oheadline

    \textbox\empty{1cm}{2cm}{\hsize-2cm}{}
        \TeX, like any programming language, provides support for conditionals.
        \TeX{} conditionals have the form

        \tokcenterline{\macrousage\ifX<condition> ... \fi/emacrousage }

        where \macroname\ifX{} is any one of the \TeX{} conditional primitives.
        \TeX{} also has support for \macroname\else s.
    \endtextbox

    \textbox\coloredbox{1cm}{\lastey+.5cm}{2cm}{{orange}}
        \centerline{\color{white}\macroname\if}
    \endtextbox

    \textbox\empty{4cm}{\lastoy}{\hsize-5cm}{}
        \macrousage\if<token1><token2>/emacrousage: \macroname\if{} expands whatever comes after it until it finds two
        unexpandable checks if both tokens have the same character code, that is they represent the same character.

        If one of the tokens is a control sequence, if it has been \macroname\let{} equal to some non-active character then it is
        considered to have the same category code and character code as that character (at the time that it was \macroname\let).
        Otherwise it is considered to have category code and character code incompatible with normal characters, but comparing
        two macros will be true.
    \endtextbox

    \textbox\coloredbox{1cm}{\lastey+.5cm}{2cm}{{orange}}
        \centerline{\color{white}\macroname\ifcat}
    \endtextbox

    \textbox\empty{4cm}{\lastoy}{\hsize-5cm}{}
        \macrousage\ifcat<token1><token2>/emacrousage: the same rules apply to \macroname\ifcat{} as \macroname\if, but the
        category codes are compared instead of the character codes.

        In order to supress the expansion of active characters, you must prepend them with \macroname\noexpand.
    \endtextbox

    \ofootline{Macros and primitives}
\endslide

\beginslide
    \oheadline

    \textbox\coloredbox{1cm}{2cm}{2cm}{{orange}}
        \centerline{\color{white}\macroname\ifx}
    \endtextbox

    \textbox\empty{4cm}{\lastoy}{\hsize-5cm}{}
        \macrousage\ifx<token1><token2>/emacrousage: the tokens are {\it not} expanded.
        \macroname\ifx{} checks that both the category code and the character code of the tokens match if they are characters (or
        \macroname\let{} equal to characters).
        If the tokens are control sequences, \macroname\ifx{} checks that they have the same expansion (if they are primitives
        then it checks that they are the same primitive).
    \endtextbox

    \textbox\coloredbox{1cm}{\lastey+.5cm}{2cm}{{orange}}
        \centerline{\color{white}\macroname\ifnum}
    \endtextbox

    \textbox\empty{4cm}{\lastoy}{\hsize-5cm}{}
        \macrousage\ifnum<num1><relation><num2>/emacrousage: compares two integers {\it num1}\hfill\break and {\it num2}.
        {\it relation} must be either {\tt=}, {\tt<}, or {\tt>} with category code $12$.
    \endtextbox

    \ofootline{Macros and primitives}
\endslide

\beginslide
    \oheadline

    \textbox\empty{1cm}{1.5cm}{\hsize-2cm}{}
        The following powerful primitives give us tools to change the flow of expansion.
    \endtextbox

    \textbox\coloredbox{1cm}{\lastey+.5cm}{4.5cm}{{orange}}
        \centerline{\color{white}\macroname\expanded}
    \endtextbox

    \textbox\empty{6cm}{\lastoy}{\hsize-7cm}{}
        \macrousage\expanded{<tokens>}/emacrousage: This will totally expand {\it tokens}, similar to \macroname\edef{} but
        \macroname\expanded{} is itself expandable.

        This is a pdf\TeX{} primitive.
    \endtextbox

    \textbox\coloredbox{1cm}{\lastey+.5cm}{4.5cm}{{orange}}
        \centerline{\color{white}\macroname\unexpanded}
    \endtextbox

    \textbox\empty{6cm}{\lastoy}{\hsize-7cm}{}
        \macrousage\expanded{<tokens>}/emacrousage: The expansion is {\it tokens}, this\hfil\break means that when placed inside
        an expansion-only environment like \macroname\expanded{}, {\it tokens} are not expanded.

        This is a $\epsilon$-\TeX{} primitive.
    \endtextbox

    \textbox\coloredbox{1cm}{\lastey+.5cm}{4.5cm}{{orange}}
        \centerline{\color{white}\macroname\futurelet}
    \endtextbox

    \textbox\empty{6cm}{\lastoy}{\hsize-7cm}{}
        \macrousage\futurelet\<macro><token1><token2>/emacrousage: This is\hfil\break effectively the same as

        \tokcenterline{\macrousage \let\<macro>=<token2><token1><token2>/emacrousage }
        This is very useful when you want to look ahead at the next token in the stream without removing it from the stream.
    \endtextbox

    \textbox\coloredbox{1cm}{\lastey+.5cm}{4.5cm}{{orange}}
        \centerline{\color{white}\macroname\aftergroup}
    \endtextbox

    \textbox\empty{6cm}{\lastoy}{\hsize-7cm}{}
        \macrousage\aftergroup<token>/emacrousage: {\it token} is inserted after the current group.
    \endtextbox

    \textbox\coloredbox{1cm}{\lastey+.5cm}{4.5cm}{{orange}}
        \centerline{\color{white}\macroname\afterassignment}
    \endtextbox

    \textbox\empty{6cm}{\lastoy}{\hsize-7cm}{}
        \macrousage\afterassignment<token>/emacrousage: {\it token} is inserted after the next assignment (\macroname\def,
        \macroname\let, a register assignment like \macroname\setbox, etc).
    \endtextbox

    \ofootline{Macros and primitives}
\endslide

\beginslide
    \oheadline

    \textbox\empty{1cm}{2cm}{\hsize-2cm}{}
        Let us now show some uses of \macroname\futurelet{} and \macroname\afterassignment.
        A good example of these primitives are the handy \LaTeX{} macros \macroname\@ifnextchar{} and \macroname\@ifstar.

        The general uses are as follows:
    \endtextbox

    \textbox\coloredbox{1cm}{\lastey+.5cm}{3.5cm}{{orange}}
        \centerline{\color{white}\macroname\@ifnextchar}
    \endtextbox

    \textbox\empty{5cm}{\lastoy}{\hsize-7cm}{}
        \macrousage\@ifnextchar<token1>{<true>}{<false>}<token2>/emacrousage: if {\it token1} and {\it token2} are equal then
        {\it true} is inserted into the sream (it is run), otherwise {\it false} is.
        This is done without removing {\it token2} from the stream.
    \endtextbox

    \textbox\coloredbox{1cm}{\lastey+.5cm}{3.5cm}{{orange}}
        \centerline{\color{white}\macroname\@ifstar}
    \endtextbox

    \textbox\empty{5cm}{\lastoy}{\hsize-7cm}{}
        \macrousage\@ifstar{<true>}{<false>}<token>/emacrousage: if {\it token} is an asterisk ({\tt*}), {\it true} is inserted
        into the sream (it is run), otherwise {\it false} is.

        As opposed to just running \icode \@ifnextchar*/eicode, the asterisk is removed in the case that {\it token} is an
        asterisk.
    \endtextbox

    \ofootline{Macros and primitves}
\endslide

\beginslide
    \oheadline

    \textbox\empty{1cm}{2cm}{\hsize-2cm}{}
        A naive implementation of 
    \endtextbox

    \ofootline{Macros and primitives}
\endslide

\bye
